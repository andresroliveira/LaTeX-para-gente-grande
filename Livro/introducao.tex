\chapter{Introdução}

\lettrine{E}{ste} livro é um guia para o minicurso ``\booktitle'' que será ministrado para o LEM - Laboratório de Ensino de Matemática do IMECC - Instituto de Matemática, Estatística e Computação Científica da UNICAMP - Universidade Estadual de Campinas. LaTeX para gente grande pressupõe que tenham conhecimento básico de LaTeX e que deseja aprimorar seus conhecimentos. 


\section{O que é \LaTeX?}

Qual a diferença entre \LaTeX\ e \TeX?

O \TeX\ é um sistema de tipografia desenvolvido por Donald Knuth no final da década de 1970, projetado para produzir documentos com alta qualidade tipográfica, especialmente textos técnicos e matemáticos. Ele é uma linguagem de baixo nível que permite controle detalhado sobre o layout do documento. Apesar de ser muito poderoso, o \TeX\ pode ser complexo e trabalhoso para usuários iniciantes, uma vez que exige a definição manual de muitos aspectos do documento.

Por outro lado, o \LaTeX, desenvolvido por Leslie Lamport nos anos 1980, é um conjunto de macros baseado no \TeX. Ele simplifica o processo de criação de documentos ao fornecer comandos de alto nível para estruturar e formatar textos, deixando os detalhes técnicos para o \TeX. O \LaTeX\ é amplamente utilizado em publicações científicas e acadêmicas devido à sua facilidade de uso e capacidade de produzir documentos consistentes e bem formatados.

\section{Por que usar \LaTeX?}

Existem várias vantagens em usar o \LaTeX\ para escrever documentos:

\begin{itemize}
    \item \textbf{Qualidade tipográfica:} O \LaTeX\ produz documentos com alta qualidade tipográfica, graças ao algoritmo de quebra de linha e ao sistema de fontes de alta qualidade.
    \item \textbf{Facilidade de formatação:} O \LaTeX\ fornece comandos simples e intuitivos para formatar textos, tais como títulos, seções, listas, tabelas e figuras.
    \item \textbf{Consistência:} O \LaTeX\ garante a consistência do layout do documento, evitando erros comuns de formatação.
    \item \textbf{Reutilização:} O \LaTeX\ permite reutilizar trechos de texto e comandos em vários documentos, facilitando a manutenção e atualização de conteúdo.
    \item \textbf{Controle detalhado:} O \LaTeX\ oferece controle detalhado sobre o layout do documento, permitindo ajustar o espaçamento, a formatação e o estilo de texto conforme necessário.
    \item \textbf{Compatibilidade:} O \LaTeX\ é compatível com vários sistemas operacionais e editores de texto, facilitando a colaboração e o compartilhamento de documentos.

\end{itemize}

\section{A estrutura de um documento \LaTeX}

Um documento \LaTeX\ é composto duas principais partes:

\begin{itemize}
    \item \textbf{Preambulo:} A parte inicial do documento que contém informações sobre o tipo de documento, pacotes utilizados, configurações de formatação e comandos personalizados.
    \item \textbf{Corpo:} A parte principal do documento que contém o texto, títulos, seções, listas, tabelas, figuras e outros elementos.
\end{itemize}


\section{Como usar este livro}

Esse livro usa o template \'e baseado no \cite{carauma}.



